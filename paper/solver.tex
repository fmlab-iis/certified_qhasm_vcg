
By Theorem~\ref{theorem:program-to-q-soundness}, it remains to show 
that 
\[
\begin{array}{l}
  \forall \vx \in \bbfZ^{|\vx|}.
  \bigwedge\limits_{i \in [I]} e_i (\vx) = e'_i (\vx) \wedge
  \bigwedge\limits_{j \in [J]} f_j (\vx) \equiv f'_j (\vx) \mod n_j
  \implies
  \\
  \hspace{.3\textwidth}
  \bigwedge\limits_{k \in [K]} g_k (\vx) = g'_k (\vx) \wedge
  \bigwedge\limits_{l \in [L]} h_l (\vx) \equiv h'_l (\vx) \mod m_l
\end{array}
\]
where
$e_i (\vx), e'_i (\vx), f_j (\vx), f'_j (\vx),
 g_k (\vx), g'_k (\vx), h_l (\vx), h'_l (\vx) \in
 \bbfZ[\vx]$, $n_j, m_l \in \bbfN$ for $i \in [I]$, $j \in [J]$, $k
 \in [K]$, and $l \in [L]$. Since the
implicant is a conjunction of (modulo) equations, it suffces to 
prove one conjunct at a time. That is, we aim to show that 
\begin{equation}
  \label{eq:polynomial-equation-implicant}
  \forall \vx \in \bbfZ^{|\vx|}.
  \bigwedge\limits_{i \in [I]} e_i (\vx) = e'_i (\vx) \wedge
  \bigwedge\limits_{j \in [J]} f_j (\vx) \equiv f'_j (\vx) \mod n_j
  \implies
  g (\vx) = g' (\vx)
\end{equation}
where $e_i (\vx), e'_i (\vx), f_j (\vx), f'_j (\vx), g (\vx), g' (\vx)
\in \bbfZ[\vx]$ for $i \in [I], j \in [J]$; or, 
 \begin{equation}
   \label{eq:modular-polynomial-equation-implicant}
   \forall \vx \in \bbfZ^{|\vx|}.
   \bigwedge\limits_{i \in [I]} e_i (\vx) = e'_i (\vx) \wedge
   \bigwedge\limits_{j \in [J]} f_j (\vx) \equiv f'_j (\vx) \mod n_j
   \implies
   h (\vx) \equiv h' (\vx) \mod m
 \end{equation}
where $e_i (\vx), e'_i (\vx), f_j (\vx), f'_j (\vx), h (\vx), h' (\vx)
\in \bbfZ[\vx]$ for $i \in [I], j \in [J]$, and $m \in \bbfN$. 

It is not hard to rewrite modular polynomial equations in antecedents
in (\ref{eq:polynomial-equation-implicant}) and 
(\ref{eq:modular-polynomial-equation-implicant}) to polynomial
equations~\cite{H:07:AENTP}. By definition of modular polynomial equations,
(\ref{eq:polynomial-equation-implicant}) is equivalent to
\[
\forall \vx \in \bbfZ^{|\vx|}.
\bigwedge\limits_{i \in [I]} e_i (\vx) = e'_i (\vx) \wedge
\bigwedge\limits_{j \in [J]} [\exists d_j. f_j (\vx) = f'_j (\vx) + d_j \cdot n_j]
\implies
g (\vx) = g' (\vx),
\]
which in turn is equivalent to
\[
\forall \vx \in \bbfZ^{|\vx|} \vd \in \bbfZ^{J}.
\bigwedge\limits_{i \in [I]} e_i (\vx) = e'_i (\vx) \wedge
\bigwedge\limits_{j \in [J]} f_j (\vx) = f'_j (\vx) + d_j \cdot n_j
\implies
g (\vx) = g' (\vx).
\]

\hide{
Consider the following system of $r + s$ (modulo) equations:
\begin{equation}
  \label{equation:modulo-polynomials}
  \begin{array}{rclcrcl}
    e_1 (\vx) & = & e'_1 (\vx) 
    & \hspace{.05\textwidth} &
    f_1 (\vx) & \equiv & f'_1 (\vx) \mod n_1\\
    & \vdots & & & & \vdots \\
    e_i (\vx) & = & e'_i (\vx) & &
    f_j (\vx) & \equiv & f'_j (\vx) \mod n_j\\
    & \vdots & & & & \vdots \\
    e_r (\vx) & = & e'_r (\vx) & &
    f_s (\vx) & \equiv & f'_s (\vx) \mod n_s
  \end{array}
\end{equation}
Our goal is to establish $g (\vx) = g' (\vx)$ or $h (\vx)
\equiv h (\vx) \mod m$ for every $\vx$ satisfying the system
of equations~(\ref{equation:modulo-polynomials}). In order to prove it
in \coq, we first rewrite modulo equations $f_j (\vx) \equiv f'_j
(\vx) \mod n_j$ to existentially quantified equations $\exists d_j
\in \bbfZ, f_j (\vx) = f'_j (\vx) + n_j \times d_j$. In other
words, we want to show that for every $\vx \in \bbfZ^{|\vx|}$
satisfying 
\begin{equation}
  \label{equation:polynomials}
  \begin{array}{rclcrcl}
    e_1 (\vx) & = & e'_1 (\vx) 
    & \hspace{.05\textwidth} &
    f_1 (\vx) & = & f'_1 (\vx) + n_1 \times d_1 \\
    & \vdots & & & & \vdots \\
    e_i (\vx) & = & e'_i (\vx) & &
    f_j (\vx) & = & f'_j (\vx) + n_j \times d_j \\
    & \vdots & & & & \vdots \\
    e_r (\vx) & = & e'_r (\vx) & &
    f_s (\vx) & = & f'_s (\vx) + n_s \times d_s
  \end{array}
\end{equation}
with some $d_1, \ldots, d_s$, then $g (\vx) = g' (\vx)$ or $h
(\vx) \equiv h' (\vx) \mod m$. 
}

In other words, it suffices to consider the following
\emph{polynomial equation entailment} problems~\cite{H:07:AENTP}:
\begin{equation}
  \label{eq:reduced-polynomial-equation-implicant}
  \forall \vx \in \bbfZ^{|\vx|}.
  \bigwedge\limits_{i \in [I]} e_i (\vx) = e'_i (\vx)
  \implies
  g (\vx) = g' (\vx)
\end{equation}
where $e_i (\vx), e'_i (\vx), g (\vx), g' (\vx) \in \bbfZ[\vx]$ for $i
\in [I]$; or,
 \begin{equation}
   \label{eq:reduced-modular-polynomial-equation-implicant}
   \forall \vx \in \bbfZ^{|\vx|}.
   \bigwedge\limits_{i \in [I]} e_i (\vx) = e'_i (\vx)
   \implies
   h (\vx) \equiv h' (\vx) \mod m
 \end{equation}
where $e_i (\vx), e'_i (\vx), h (\vx), h' (\vx) \in \bbfZ[\vx]$ for $i
\in [I]$ and $m \in \bbfN$.

One way to solve the polynomial equation entailment problems
~(\ref{eq:reduced-polynomial-equation-implicant})
and~(\ref{eq:reduced-modular-polynomial-equation-implicant}) is by
solving the ideal membership problem~\cite{H:07:AENTP,BS:16:GFEV}. 
For~(\ref{eq:reduced-polynomial-equation-implicant}), consider the
ideal $I = \langle e_i(\vx) - e'_i(\vx) \rangle_{i \in [I]}$. Suppose
$g(\vx) - g'(\vx) \in I$. Then there are $u_i(\vx) \in \bbfZ[\vx]$
such that 
\[
g(\vx) - g'(\vx) = \sum\limits_{i \in [I]} u_i (\vx) [e_i (\vx) - e' (\vx)].
\]
Hence $g(\vx) - g'(\vx) = 0$ by the polynomial equations $e_i (\vx) =
e' (\vx)$ for $i \in [I]$ in the antecedents
in~(\ref{eq:reduced-polynomial-equation-implicant}). Similarly, it
suffices to check if $h(\vx) - h'(\vx) \in \langle m, e_i(\vx) -
e'_i(\vx) \rangle_{i \in [I]}$
for~(\ref{eq:reduced-modular-polynomial-equation-implicant}).
The reduction to the ideal membership problem however is sound but
incomplete. There are instances
of~(\ref{eq:reduced-polynomial-equation-implicant})
or~(\ref{eq:reduced-modular-polynomial-equation-implicant}) whose
corresponding ideal membership problems do not hold. 

The ideal membership problem can be solved by finding a Gr\"obner
basis for the ideal. There are two \coq tactics to
construct formal proofs for the polynomial equation entailment
problems~(\ref{eq:reduced-polynomial-equation-implicant})
and~(\ref{eq:reduced-modular-polynomial-equation-implicant})~\cite{P:08:CGBP,P:10:CGBP}. 
The tactic \dslcode{nsatz} is able to prove
equalities in integral domains. The tactic \dslcode{gbarith} is able
to prove modulo equalities in integral domains. We define a
\coq tactic which invokes \dslcode{nsatz} or \dslcode{gbarith} 
to prove $g (\vx) = g' (\vx)$ or $h (\vx) \equiv
h' (\vx) \mod m$ respectively. Finding Gr\"obner bases for ideals is
known to be an NP-complete problem. \todo{reference?} 
The tactics \dslcode{nsatz} and \dslcode{gbarith} are not very
scalable.
For modulo arithmetic over large finite fields, their 
implementations can have hundreds of polynomial equations
in~(\ref{eq:reduced-polynomial-equation-implicant})
or~(\ref{eq:reduced-modular-polynomial-equation-implicant}). 
The \coq tactics are unable to finish
the proof in a reasonable amount of time. A more scalable tactic is
necessary to verify implementations of low-level cryptographic codes.

For the polynomial equations generated by \textit{ProgramToQ}
($p$) (Algorithm~\ref{algorithm:polynomial-programs}), they are of the
following forms: $x = E$ (from an assignment statement) or $x + 2^c y
= E$ (from a \dslcode{Split} statement). Such polynomial equations can
safely be removed after every occurrences of $x$ are replaced with $E$
or $E - 2^c y$. The number of polynomial equations is thus greatly
reduced. We hence define a \coq tactic to rewrite variables and then
remove corresponding polynomial equations. For proving a polynomial
equation $g (\vx) = g' (\vx)$, it turns out that the \coq tactic
\dslcode{ring} suffices. For a modulo equation $h (\vx) \equiv
h' (\vx) \mod m$, \todo{we need what?}
