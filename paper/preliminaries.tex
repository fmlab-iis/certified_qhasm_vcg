
We write $\mathbb{B} = \{ \mathit{ff}, \mathit{tt} \}$ for the Boolean
domain. Let $\bbfN$ and $\bbfZ$ denote positive and all integers
respectively. We use $[n]$ to denote the set $\{ 1, 2, \ldots, n \}$
for $n \in \bbfN$. 

A \emph{group} $\mathcal{G} = (G, 0, +)$ is an algebraic structure
where $G$ is a set, $0 \in G$, and $+$ is an associative binary
operator over $G$ such that $0 + a = a + 0 = a$ and there is a $-a \in
G$ such that $(-a) + a = a + (-a) = 0$ for every $a \in G$. The
element $0$ is the \emph{identity} of the group; the element $-a$ is
called the \emph{inverse} of $a$. $\mathcal{G}$ is \emph{Abelian} if
the operator $+$ is commutative.
A \emph{ring} $\mathcal{R} = (R, 0, 1, +, \times)$ consists of a set
$R$, $0$, $1 \in R$ with $0 \neq 1$, and $+$, $\times$ binary
functions over $R$ that
\begin{itemize}
\item $(R, 0, +)$ is an abelian group; 
\item $(R \setminus \{ 0 \}, 1, \times)$ is a group; and 
\item $\times$ is distributive over $+$: $a \times (b + c) = a \times
  b + a \times c$ for every $a, b, c \in R$.
\end{itemize}
If $\times$ is commutative, the ring $\mathcal{R}$ is
\emph{commutative}. $a \in R$ is a \emph{zero divisor} if 
$a \times b = 0$ for some $b \in R \setminus \{ 0 \}$. An
\emph{integral domain} is a commutative ring without zero divisors. 
A \emph{field} is a commutative integral domain. $\bbfN$ is not a ring.
$\bbfZ$ is an integral domain but not a field.
For any prime number $\varrho$, $\bbfGF(\varrho)$
is the Galois field of order $\varrho$. We focus on Galois fields of
very large orders, say, $\myprime = 2^{255} - 19$. 

Fix a set of variables $\vx$. $\bbfF[\vx]$ is the set of multivariant
polynomials over $\vx$ with coefficients in $\bbfF$. A set 
$I \subseteq \bbfF[\vx]$ is an \emph{ideal} if
\begin{itemize}
\item $f +_\bbfF g \in I$ for every $f, g \in I$; and
\item $h \cdot_\bbfF f \in I$ for every $h \in \bbfF[\vx]$ and $f \in I$.
\end{itemize}
We write $\langle G \rangle$ for the minimal ideal containing $G
\subseteq \bbfF[\vx]$. Given an ideal $I$ and a polynomial $f \in
\bbfF[\vx]$, the \emph{ideal membership} problem is to decide whether
$f \in I$.