
We write $\mathbb{B} = \{ \mathit{ff}, \mathit{tt} \}$ for the Boolean
domain. Let $\bbfN$ and $\bbfZ$ denote positive and all integers
respectively. We use $[n]$ to denote the set $\{ 1, 2, \ldots, n \}$
for $n \in \bbfN$. For any prime number $\varrho$, $\bbfGF(\varrho)$
is the Galois field of order $\varrho$. We focus on Galois fields of
very large orders, say, $\myprime = 2^{255} - 19$. 

A {\em ring} $\mathcal{R} = (R, 0, 1, +, \times)$ consists of a set
$R$, $0$, $1 \in R$ with $0 \neq 1$, and $+$, $\times$ binary
functions over $R$ that
\begin{itemize}
\item $(R, 0, +)$ is an abelian group: $(a + b) + c = a + (b
  + c)$, $a + b = b + a$, $a + 0 = a$, and $a + (-a) = 0$ for every
  $a, b, c \in R$; 
\item $(R, 1, \times)$ is a group: $(a \times b) \times c = a
  \times (b \times c)$, $a \times 1 = 1 \times a = a$, $a \times
  a^{-1} = a^{-1} \times a = 1$ for every $a, b, c \in R$; and 
\item $\times$ is distributive over $+$: $a \times (b + c) = a \times
  b + a \times c$ for every $a, b, c \in R$.
\end{itemize}
If $\times$ is commutative, the ring $\mathcal{R}$ is
\emph{commutative}. $a \in R$ is a \emph{zero divisor} if there is an
$x \in R$ such that $a \times x = 0$. An \emph{integral domain} is a
commutative ring without zero divisors. $\bbfN$ is not a ring.
$\bbfZ$ is an integral domain.

Fix a set of variables $\vx$. $\bbfF[\vx]$ is the set of multivariant
polynomials over $\vx$ with coefficients in $\bbfF$. A set 
$I \subseteq \bbfF[\vx]$ is an \emph{ideal} if
\begin{itemize}
\item $f +_\bbfF g \in I$ for every $f, g \in I$; and
\item $h \cdot_\bbfF f \in I$ for every $h \in \bbfF[\vx]$ and $f \in I$.
\end{itemize}
We write $\langle G \rangle$ for the minimal ideal containing $G
\subseteq \bbfF[\vx]$. Given an ideal $I$ and a polynomial $f \in
\bbfF[\vx]$, the \emph{ideal membership} problem is to decide whether
$f \in I$.