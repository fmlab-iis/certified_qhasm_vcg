\documentclass{llncs}

\usepackage{amsmath}
\usepackage{amsfonts}
\usepackage{amssymb}
%\usepackage{listings}
\usepackage[usenames,dvipsnames]{color}
\usepackage{xspace}
\usepackage{multirow}
\pagestyle{plain}
%\usepackage{graphicx}
\usepackage{algorithm,algpseudocode}
%\usepackage[T1]{fontenc} 
\renewcommand{\algorithmicrequire}{\textbf{Input:}}
\renewcommand{\algorithmicensure}{\textbf{Output:}}

\usepackage{a4wide}


\newcommand{\todo}[1]{\textcolor{blue}{TODO: {#1}}}

\newcommand{\defn}{\triangleq}
\newcommand{\dslcode}[1]{\mbox{\textsf{#1}}}
\newcommand{\code}[1]{\mbox{#1}}
\newcommand{\uscore}[0]{{\char95}}
\newcommand{\cond}[1]{{(\!|}{#1}{|\!)}}
\newcommand{\valueof}[1]{[\!|{#1}|\!]}
\newcommand{\bbfN}{\mathbb{N}}
\newcommand{\bbfZ}{\mathbb{Z}}
\newcommand{\bbfF}{\mathbb{F}}
\newcommand{\bbfGF}{\mathbb{GF}}
\newcommand{\coq}{\textsc{Coq}\xspace}
\newcommand{\vx}{\vec{x}}
\newcommand{\myprime}{\varrho}
\newcommand{\hoaretriple}[3]{\cond{#1}~{#2}~\cond{#3}}

\algnewcommand\algorithmicmatch{\textbf{match}}
\algnewcommand\algorithmicwith{\textbf{with}}
\algnewcommand\algorithmiccase{\textbf{case}}
% New "environments"
\algdef{SE}[MATCH]{Match}{EndMatch}[1]{\algorithmicmatch\ {#1}\ \algorithmicwith}{\algorithmicend\ \algorithmicmatch}%
\algdef{SE}[CASE]{Case}{EndCase}[1]{\algorithmiccase\ {#1}:}{\algorithmicend\ \algorithmiccase}%
\algtext*{EndMatch}%
\algtext*{EndCase}%
\algnewcommand{\IfThenElse}[3]{% \IfThenElse{<if>}{<then>}{<else>}
  \State \algorithmicif\ #1\ \algorithmicthen\ #2\ \algorithmicelse\ #3}


\usepackage{xcolor}
\definecolor{linkcolor}{rgb}{0.65,0,0}
\definecolor{citecolor}{rgb}{0,0.65,0}
\definecolor{urlcolor}{rgb}{0,0,0.65}
\usepackage[colorlinks=true, backref=page, linkcolor=linkcolor, urlcolor=urlcolor, citecolor=citecolor]{hyperref}

\newif\ifpublic
%\publictrue

\title{Certified Verification of Algebraic Properties on
Low-Level Mathematical Constructs in Cryptographic Programs} 

\ifpublic
\author{
Ming-Hsien Tsai
\and
Bow-Yaw Wang
\and
Bo-Yin Yang
%
%\thanks{}
}
\institute
{
Institute of Information Science\\
Academia Sinica\\
128 Section 2 Academia Road, Taipei 115-29, Taiwan\\
\email{mhtsai208@gmail.com, bywang@iis.sinica.edu.tw, by@crypto.tw}
}
\else
\author{\vspace*{0cm}}
\institute{\vspace*{0cm}\ }
\fi

\begin{document}

\maketitle

\begin{abstract}
  Mathematical constructs are needed to perform algebraic operations
  on the underlying mathematical structures of cryptosystems. 
  Since they are invoked most frequently, these mathematical constructs
  are written in assembly languages and often optimized manually. We
  propose a fully certified technique to formally verify mathematical
  constructs in X25519, the default elliptic curve Diffie-Hellman key
  exchange protocol used in \openssh. Our technique first translates any
  algebraic specification of mathematical constructs into an algebraic
  problem. The algebraic 
  problem in turn is solved by \coq and \singular. The proof assistant 
  \coq is used to certify the translation and solution to algebraic
  problems. We report our experiments on the verification of
  arithmetic computation over the Galois field used in X25519, where
  each element of the field is represented by a 255-bit number. We
  also verify the Montgomery Ladderstep used in the key exchange
  protocol.
\end{abstract}

\section{Introduction}
\label{section:introduction}

\todo{Bo-Yin, please motivate why the correctness of Montgomery ladder
step in Curve25519 is important for slightly over .5 page.}

We have the following contributions:
\begin{itemize}
\item We propose a domain specific language for modeling low-level
  cryptographic programs over large finite fields.
\item We give a certified verification condition generator from
  programs written in our domain specific language to polynomial
  (modulo) equations over integral domains.
\item We verify arithmetic computation over a large finite field and a
  critical program (the Montgomery ladder step) automatically in a
  reasonable amount of time.  
\item To the best of our knowledge, our work is the first automatic
  and certified verification on real cryptographic programs with
  minimal human intervention.
\end{itemize}


\section{Preliminaries}
\label{section:preliminaries}



\section{Domain Specific Language -- \mydsl}
\label{section:domain-specific-language}

\begin{eqnarray*}
  N & ::= & \dslcode{1}\ |\ \dslcode{2}\ |\ \cdots\\
  C & ::= & \cdots \ |\ \dslcode{-2}\ |\ \dslcode{-1} \ |\ 0\ |\ 
            \dslcode{1}\ |\ \dslcode{2}\ |\ \cdots\\
  V & ::= & \dslcode{x} \ |\ \dslcode{y} \ |\ \dslcode{z} \ |\ \cdots\\
  E & ::= &  C \ |\ V \ |\  \dslcode{-}E \ |\ E \dslcode{+} E 
             \ |\ E \dslcode{-} E
             \ |\ E \times E \ |\ \dslcode{Pow} (E, N)
\end{eqnarray*}

\begin{eqnarray*}
  S & ::= & V \leftarrow E 
            \ |\  V, V \leftarrow \dslcode{Split} (E, N)\\
  P & ::= & \epsilon \ |\ S; P
\end{eqnarray*}

\begin{eqnarray*}
  Q & ::= & \top
     \ |\   E = E
     \ |\   E \equiv E \mod N
     \ |\   Q \wedge Q
\end{eqnarray*}

\begin{eqnarray*}
  H & ::= & \cond{Q} P \cond{Q}
\end{eqnarray*}

\section{Translation to Algebraic Problems}
\label{section:translation}

\subsection{Slicing}
\label{subsection:slicing}

\begin{algorithm}
  \begin{algorithmic}[1]
    \Function{VarsInE}{$e$}
    \Match{$e$}
      \Case{$V$}
        \Return $\{ e \}$
      \EndCase
      \Case{$Z$}
        \Return $\emptyset$
      \EndCase
      \Case{$-e'$}
        \Return \Call{VarsInE}{$e'$}
      \EndCase
      \Case{$e' + e''$}
        \Return \Call{VarsInE}{$e'$} $\cup$ \Call{VarsInE}{$e''$}
      \EndCase
      \Case{$e' - e''$}
        \Return \Call{VarsInE}{$e'$} $\cup$ \Call{VarsInE}{$e''$}
      \EndCase
      \Case{$e' \times e''$}
        \Return \Call{VarsInE}{$e'$} $\cup$ \Call{VarsInE}{$e''$}
      \EndCase
      \Case{$\dslcode{Pow}(e', \uscore)$}
        \Return \Call{VarsInE}{$e'$}
      \EndCase
    \EndMatch
    \EndFunction
  \end{algorithmic}
  \caption{Variables Occurred in an Expression}
\end{algorithm}

\begin{algorithm}
\begin{algorithmic}[1]
  \Function{VarsInQ}{$q$}
  \Match{$q$}
    \Case{$\top$}
      $\emptyset$
    \EndCase
    \Case{$e' = e''$}
      \Return \Call{VarsInE}{$e'$} $\cup$ \Call{VarsInE}{$e''$}
    \EndCase
    \Case{$e' \equiv e'' \mod \uscore$}
      \Return \Call{VarsInE}{$e'$} $\cup$ \Call{VarsInE}{$e''$}
    \EndCase
    \Case{$q' \wedge q''$}
      \Return \Call{VarsInQ}{$q'$} $\cup$ \Call{VarsInQ}{$q''$}
    \EndCase
  \EndMatch
  \EndFunction
\end{algorithmic}
\caption{Variables Occurred in Predicates}
\end{algorithm}

\begin{algorithm}
  \begin{algorithmic}[1]
    \Function{SliceStatement}{$\mathit{vars}$, $s$}
    \Match{$s$}
      \Case{$v \leftarrow e$}
        \IfThenElse{$v \in \mathit{vars}$}
          {\Return \Call{VarsInE}{$e$} $\cup$ $(\mathit{vars} \setminus \{
            v \})$, $s$}
          {\Return $\mathit{vars}$}
      \EndCase
      \Case{$v_h, v_l \leftarrow \dslcode{Split} (e, \uscore)$}
        \IfThenElse{$v_h \in \mathit{vars}$ \textbf{or} $v_l \in \mathit{vars}$}
          {\Return \Call{VarsInE}{$e$} $\cup$ $(\mathit{vars} \setminus \{
            v_h, v_h \})$, $s$}
          {\Return $\mathit{vars}$}
      \EndCase
    \EndMatch
    \EndFunction
  \end{algorithmic}
  \caption{Slicing a Statement}
\end{algorithm}

\begin{algorithm}
  \begin{algorithmic}[1]
    \Function{SliceProgram}{$\mathit{vars}$, $pp$}
    \Match{$p$}
      \Case{$\epsilon$}
        \Return $\epsilon$
      \EndCase
      \Case{$pp; s;$}
        \Match{\Call{SliceStatement}{$\mathit{vars}$, $s$}}
          \Case{$\mathit{vars}'$}
            \Return \Call{SliceProgram}{$\mathit{vars}'$, $pp$}
          \EndCase
          \Case{$\mathit{vars}'$, $s'$}
            \Return \Call{SliceProgram}{$\mathit{vars}'$, $pp$}$\ s';$
          \EndCase
        \EndMatch
      \EndCase
    \EndMatch
    \EndFunction
  \end{algorithmic}
  \caption{Slicing a Program}
\end{algorithm}

\begin{theorem}
  For every $q, q' \in Q$ and $p \in P$,
  \begin{center}
  $\models \hoaretriple{q}{p}{q'}$ if and only if
  $\models \hoaretriple{q}{\textsc{SliceProgram}(\textsc{VarsInQ}(q'), p)}
  {q'}$.
  \end{center}

\end{theorem}


\subsection{Static Single Assignment}
\label{subsection:static-single-assignment}

\begin{definition}
  A program is in \emph{single static assignment} form if every
  variable is never defined twice (or more).
\end{definition}

\begin{algorithm}
  \begin{algorithmic}[1]
    \Function{SSAE}{$\theta$, $e$}
    \Match{$e$}
      \Case{$V$} \Return $e^{\theta(e)}$ \EndCase
      \Case{$Z$} \Return $e$ \EndCase
      \Case{$-e'$} \Return $-$\Call{SSAE}{$\theta$, $e'$} \EndCase
      \Case{$e' + e''$} 
        \Return \Call{SSAE}{$\theta$, $e'$} $+$ 
                \Call{SSAE}{$\theta$, $e''$}
      \EndCase
      \Case{$e' - e''$} 
        \Return \Call{SSAE}{$\theta$, $e'$} $-$ 
                \Call{SSAE}{$\theta$, $e''$}
      \EndCase
      \Case{$e' \times e''$} 
        \Return \Call{SSAE}{$\theta$, $e'$} $\times$ 
                \Call{SSAE}{$\theta$, $e''$}
      \EndCase
      \Case{$\dslcode{Pow}$($e'$, $n$)}
        \Return $\dslcode{Pow}$(\Call{SSAE}{$\theta$, $e'$}, $n$)
      \EndCase
    \EndMatch
    \EndFunction
  \end{algorithmic}
  \caption{Single Static Assignment Transformation for Expressions}
\end{algorithm}

\begin{algorithm}
  \begin{algorithmic}[1]
    \Function{SSAQ}{$\theta$, $q$}
    \Match{$q$}
      \Case{$\top$} \Return $\top$ \EndCase
      \Case{$e' = e''$} 
        \Return \Call{SSAE}{$\theta$, $e$} = \Call{SSAE}{$\theta$, $e'$}
      \EndCase
      \Case{$e' \equiv e'' \mod n$} 
        \Return \Call{SSAE}{$\theta$, $e$} $\equiv$ 
                \Call{SSAE}{$\theta$, $e'$} $\mod n$
      \EndCase
      \Case{$q' \wedge q''$}
        \Return \Call{SSAQ}{$\theta$, $q'$} $\wedge$
                \Call{SSAQ}{$\theta$, $q''$}
      \EndCase
    \EndMatch
    \EndFunction
  \end{algorithmic}
  \caption{Single Static Assignment Transformation for Predicates}
\end{algorithm}

\begin{algorithm}
  \begin{algorithmic}[1]
    \Function{SSAStatement}{$\theta$, $s$}
    \Match{$s$}
      \Case{$v \leftarrow e$}
        \Return $\theta[v \mapsto \theta(v) + 1]$, 
                $v^{\theta(v) + 1} \leftarrow$ \Call{SSAE}{$\theta$, $e$}
      \EndCase
      \Case{$v_h, v_l \leftarrow \dslcode{Split}(e, n)$}
        \State{\Return $\theta[v_h \mapsto \theta(v_h) + 1]
                       [v_l \mapsto \theta(v_l) + 1]$,
                $v_h^{\theta(v_h) + 1}, v_l^{\theta(v_l) + 1} \leftarrow
                \dslcode{Split}($\Call{SSAE}{$\theta$, $e$}$, n)$}
      \EndCase
    \EndMatch
    \EndFunction
  \end{algorithmic}
  \caption{Single Static Assignement Transformation for Statements}
\end{algorithm}

\begin{algorithm}
  \begin{algorithmic}[1]
    \Function{SSAProgram}{$\theta$, $p$}
    \Match{$p$}
      \Case{$\epsilon$}
        \Return $\theta$, $\epsilon$
      \EndCase
      \Case{$s; pp$}
        \State{$\theta', s' \leftarrow$ 
                 \Call{SSAStatement}{$\theta$, $s$}}
        \State{$\theta'', pp'' \leftarrow$ 
                 \Call{SSAProgram}{$\theta'$, $pp$}}
        \State{\Return $\theta''$, $s'; pp''$}
      \EndCase
    \EndMatch
    \EndFunction
  \end{algorithmic}
  \caption{Single Static Assignment for Programs}
\end{algorithm}

\begin{theorem}
  Let $\theta_0(v) = 0$ for every $v \in V$ and $p \in P$.
  If $\hat{\theta}, \hat{p} = \textsc{SSAProgram}(\theta_0, p)$, then
  $\hat{p}$ is in single static assignment form.
\end{theorem}

\begin{theorem}
  Let $\theta_0(v) = 0$ for every $v \in V$. For every $q, q' \in Q$
  and $p \in P$,
  \begin{center}
    $\cond{q}$ $p$ $\cond{q'}$ if and only if
    $\cond{\textmd{\textit{SSAQ}}(\theta_0, q)}$
    $\hat{p}$
    $\cond{\textmd{\textit{SSAQ}}(\hat{\theta}, q')}$
  \end{center}
  where $\hat{\theta}, \hat{p} =
  \textmd{\textit{SSAProgram}}(\theta_0, p)$.
\end{theorem}


\subsection{Multivariant Polynomial Equations}
\label{subsection:multivariant-polynomial-equations}

The last step transforms any program specification to the 
modular polynomial equation entailment problem defined as follows. For
$q \in Q$, we write $q(\vx)$ to signify the free variables $\vx$
occurred in $q$. Given $q(\vx), q'(\vx) \in Q$, the \emph{modular
  polynomial equation entailment} problem decides whether $\forall
\vx. q(\vx) \implies q'(\vx)$ holds for all $\vx \in \bbfZ^{|\vx|}$.

Programs in single static assignment form can easily be transformed to
conjunctions of polynomial equations. We first show how to transform
statements into polynomial equations. An assignment statement is
translated to a polynomial equation with a single variable on the left
hand side. For a $\dslcode{Split}$ statement, it is transformed to an
equation with a linear expression of the assigned variables on the
left hand side (Algorithm~\ref{algorithm:polynomial-statements}). 
\begin{algorithm}
  \begin{algorithmic}[1]
    \Function{StatementToPolyEQ}{$s$}
    \Match{$s$}
      \Case{$v \leftarrow e$}
        \Return $v = e$
      \EndCase
      \Case{$\concat{v_h}{v_l} \leftarrow \dslcode{Split}(e, n)$}
        \Return $v_l + 2^n v_h = e$
      \EndCase
    \EndMatch
    \EndFunction
  \end{algorithmic}
  \caption{Polynomial Equation Transformation for Statements}
  \label{algorithm:polynomial-statements}
\end{algorithm}

A program is transformed to the conjunction of polynomial
equations corresponding to its statements
(Algorithm~\ref{algorithm:polynomial-programs}). Both algorithms are
specified straightforwardly in \coq.

\begin{algorithm}
  \begin{algorithmic}[1]
    \Function{ProgramToPolyEQ}{$p$}
    \Match{$p$}
      \Case{$\epsilon$} \Return $\top$ \EndCase
      \Case{$s; pp$}
        \Return \Call{StatementToPolyEQ}{$s$} $\wedge$
                \Call{ProgramToPolyEQ}{$pp$}
      \EndCase
    \EndMatch
    \EndFunction
  \end{algorithmic}
  \caption{Polynomial Equation Transformation for Programs}
  \label{algorithm:polynomial-programs}
\end{algorithm}

\begin{theorem}
  Let $p \in P$ be a well-formed program in static single assignment
  form. 
\todo{Explain polynomial equation is an abstraction. It admits all
  program behaviors and definitely more.}
\end{theorem}


Note that $\textsc{ProgramToPolyEQ}(p) \in Q$ for every $p \in P$.
Definition~\ref{definition:program-entailment} gives the modular
polynomial equation entailment problem corresponding to a program
specification.
\begin{definition}
  For any $q, q' \in Q$ and $p \in P$ in single static assignment
  form, define
  \begin{eqnarray*}
    \Pi(\hoaretriple{q}{p}{q'}) & \defn &
    \forall \vx \in \bbfZ^{|\vx|}. q(\vx) \wedge \varphi(\vx) \implies q'(\vx)
  \end{eqnarray*}
  where $\varphi(\vx) =
  \textsc{ProgramToPolyEQ}(p)$. 
  %$\Pi(\hoaretriple{q}{p}{q'})$ is called the modular polynomial
  %equation entailment problem for $\hoaretriple{q}{p}{q'}$. 
  \label{definition:program-entailment}
\end{definition}

Algorithm~\ref{algorithm:polynomial-programs} is only sound for
well-formed programs. 
Theorem~\ref{theorem:program-to-q-soundness} establishes the
soundnesss of our last transformation. Its proof is also certified in
\coq. 
\begin{theorem}
  \label{theorem:program-to-q-soundness}
  Let $q, q' \in Q$ be predicates, and $p \in P$ a program in single
  static assignment form and well-formed. 
  If $\Pi(\hoaretriple{q}{p}{q'})$ holds, then
  $\models$ $\hoaretriple{q}{p}{q'}$.
\end{theorem}





\subsection{Program Slicing}
\label{subsection:translation:slicing}

\begin{algorithm}
  \begin{algorithmic}[1]
    \Function{VarsInE}{$e$}
    \Match{$e$}
      \Case{$V$}
        \Return $\{ e \}$
      \EndCase
      \Case{$Z$}
        \Return $\emptyset$
      \EndCase
      \Case{$-e'$}
        \Return \Call{VarsInE}{$e'$}
      \EndCase
      \Case{$e' + e''$}
        \Return \Call{VarsInE}{$e'$} $\cup$ \Call{VarsInE}{$e''$}
      \EndCase
      \Case{$e' - e''$}
        \Return \Call{VarsInE}{$e'$} $\cup$ \Call{VarsInE}{$e''$}
      \EndCase
      \Case{$e' \times e''$}
        \Return \Call{VarsInE}{$e'$} $\cup$ \Call{VarsInE}{$e''$}
      \EndCase
      \Case{$\dslcode{Pow}(e', \uscore)$}
        \Return \Call{VarsInE}{$e'$}
      \EndCase
    \EndMatch
    \EndFunction
  \end{algorithmic}
  \caption{Variables Occurred in an Expression}
\end{algorithm}

\begin{algorithm}
\begin{algorithmic}[1]
  \Function{VarsInQ}{$q$}
  \Match{$q$}
    \Case{$\top$}
      $\emptyset$
    \EndCase
    \Case{$e' = e''$}
      \Return \Call{VarsInE}{$e'$} $\cup$ \Call{VarsInE}{$e''$}
    \EndCase
    \Case{$e' \equiv e'' \mod \uscore$}
      \Return \Call{VarsInE}{$e'$} $\cup$ \Call{VarsInE}{$e''$}
    \EndCase
    \Case{$q' \wedge q''$}
      \Return \Call{VarsInQ}{$q'$} $\cup$ \Call{VarsInQ}{$q''$}
    \EndCase
  \EndMatch
  \EndFunction
\end{algorithmic}
\caption{Variables Occurred in Predicates}
\end{algorithm}

\begin{algorithm}
  \begin{algorithmic}[1]
    \Function{SliceStatement}{$\mathit{vars}$, $s$}
    \Match{$s$}
      \Case{$v \leftarrow e$}
        \IfThenElse{$v \in \mathit{vars}$}
          {\Return \Call{VarsInE}{$e$} $\cup$ $(\mathit{vars} \setminus \{
            v \})$, $s$}
          {\Return $\mathit{vars}$}
      \EndCase
      \Case{$v_h, v_l \leftarrow \dslcode{Split} (e, \uscore)$}
        \IfThenElse{$v_h \in \mathit{vars}$ \textbf{or} $v_l \in \mathit{vars}$}
          {\Return \Call{VarsInE}{$e$} $\cup$ $(\mathit{vars} \setminus \{
            v_h, v_h \})$, $s$}
          {\Return $\mathit{vars}$}
      \EndCase
    \EndMatch
    \EndFunction
  \end{algorithmic}
  \caption{Slicing a Statement}
\end{algorithm}

\begin{algorithm}
  \begin{algorithmic}[1]
    \Function{SliceProgram}{$\mathit{vars}$, $pp$}
    \Match{$p$}
      \Case{$\epsilon$}
        \Return $\epsilon$
      \EndCase
      \Case{$pp; s;$}
        \Match{\Call{SliceStatement}{$\mathit{vars}$, $s$}}
          \Case{$\mathit{vars}'$}
            \Return \Call{SliceProgram}{$\mathit{vars}'$, $pp$}
          \EndCase
          \Case{$\mathit{vars}'$, $s'$}
            \Return \Call{SliceProgram}{$\mathit{vars}'$, $pp$}$\ s';$
          \EndCase
        \EndMatch
      \EndCase
    \EndMatch
    \EndFunction
  \end{algorithmic}
  \caption{Slicing a Program}
\end{algorithm}

\begin{theorem}
  For every $q, q' \in Q$ and $p \in P$,
  \begin{center}
  $\models \hoaretriple{q}{p}{q'}$ if and only if
  $\models \hoaretriple{q}{\textsc{SliceProgram}(\textsc{VarsInQ}(q'), p)}
  {q'}$.
  \end{center}

\end{theorem}


\subsection{Static Single Assignments}
\label{subsection:translation:static-single-assignment}

\begin{definition}
  A program is in \emph{single static assignment} form if every
  variable is never defined twice (or more).
\end{definition}

\begin{algorithm}
  \begin{algorithmic}[1]
    \Function{SSAE}{$\theta$, $e$}
    \Match{$e$}
      \Case{$V$} \Return $e^{\theta(e)}$ \EndCase
      \Case{$Z$} \Return $e$ \EndCase
      \Case{$-e'$} \Return $-$\Call{SSAE}{$\theta$, $e'$} \EndCase
      \Case{$e' + e''$} 
        \Return \Call{SSAE}{$\theta$, $e'$} $+$ 
                \Call{SSAE}{$\theta$, $e''$}
      \EndCase
      \Case{$e' - e''$} 
        \Return \Call{SSAE}{$\theta$, $e'$} $-$ 
                \Call{SSAE}{$\theta$, $e''$}
      \EndCase
      \Case{$e' \times e''$} 
        \Return \Call{SSAE}{$\theta$, $e'$} $\times$ 
                \Call{SSAE}{$\theta$, $e''$}
      \EndCase
      \Case{$\dslcode{Pow}$($e'$, $n$)}
        \Return $\dslcode{Pow}$(\Call{SSAE}{$\theta$, $e'$}, $n$)
      \EndCase
    \EndMatch
    \EndFunction
  \end{algorithmic}
  \caption{Single Static Assignment Transformation for Expressions}
\end{algorithm}

\begin{algorithm}
  \begin{algorithmic}[1]
    \Function{SSAQ}{$\theta$, $q$}
    \Match{$q$}
      \Case{$\top$} \Return $\top$ \EndCase
      \Case{$e' = e''$} 
        \Return \Call{SSAE}{$\theta$, $e$} = \Call{SSAE}{$\theta$, $e'$}
      \EndCase
      \Case{$e' \equiv e'' \mod n$} 
        \Return \Call{SSAE}{$\theta$, $e$} $\equiv$ 
                \Call{SSAE}{$\theta$, $e'$} $\mod n$
      \EndCase
      \Case{$q' \wedge q''$}
        \Return \Call{SSAQ}{$\theta$, $q'$} $\wedge$
                \Call{SSAQ}{$\theta$, $q''$}
      \EndCase
    \EndMatch
    \EndFunction
  \end{algorithmic}
  \caption{Single Static Assignment Transformation for Predicates}
\end{algorithm}

\begin{algorithm}
  \begin{algorithmic}[1]
    \Function{SSAStatement}{$\theta$, $s$}
    \Match{$s$}
      \Case{$v \leftarrow e$}
        \Return $\theta[v \mapsto \theta(v) + 1]$, 
                $v^{\theta(v) + 1} \leftarrow$ \Call{SSAE}{$\theta$, $e$}
      \EndCase
      \Case{$v_h, v_l \leftarrow \dslcode{Split}(e, n)$}
        \State{\Return $\theta[v_h \mapsto \theta(v_h) + 1]
                       [v_l \mapsto \theta(v_l) + 1]$,
                $v_h^{\theta(v_h) + 1}, v_l^{\theta(v_l) + 1} \leftarrow
                \dslcode{Split}($\Call{SSAE}{$\theta$, $e$}$, n)$}
      \EndCase
    \EndMatch
    \EndFunction
  \end{algorithmic}
  \caption{Single Static Assignement Transformation for Statements}
\end{algorithm}

\begin{algorithm}
  \begin{algorithmic}[1]
    \Function{SSAProgram}{$\theta$, $p$}
    \Match{$p$}
      \Case{$\epsilon$}
        \Return $\theta$, $\epsilon$
      \EndCase
      \Case{$s; pp$}
        \State{$\theta', s' \leftarrow$ 
                 \Call{SSAStatement}{$\theta$, $s$}}
        \State{$\theta'', pp'' \leftarrow$ 
                 \Call{SSAProgram}{$\theta'$, $pp$}}
        \State{\Return $\theta''$, $s'; pp''$}
      \EndCase
    \EndMatch
    \EndFunction
  \end{algorithmic}
  \caption{Single Static Assignment for Programs}
\end{algorithm}

\begin{theorem}
  Let $\theta_0(v) = 0$ for every $v \in V$ and $p \in P$.
  If $\hat{\theta}, \hat{p} = \textsc{SSAProgram}(\theta_0, p)$, then
  $\hat{p}$ is in single static assignment form.
\end{theorem}

\begin{theorem}
  Let $\theta_0(v) = 0$ for every $v \in V$. For every $q, q' \in Q$
  and $p \in P$,
  \begin{center}
    $\cond{q}$ $p$ $\cond{q'}$ if and only if
    $\cond{\textmd{\textit{SSAQ}}(\theta_0, q)}$
    $\hat{p}$
    $\cond{\textmd{\textit{SSAQ}}(\hat{\theta}, q')}$
  \end{center}
  where $\hat{\theta}, \hat{p} =
  \textmd{\textit{SSAProgram}}(\theta_0, p)$.
\end{theorem}


\subsection{Modular Polynomial Equation Entailment}
\label{subsection:translation:multivariant-polynomial-equations}

The last step transforms any program specification to the 
modular polynomial equation entailment problem defined as follows. For
$q \in Q$, we write $q(\vx)$ to signify the free variables $\vx$
occurred in $q$. Given $q(\vx), q'(\vx) \in Q$, the \emph{modular
  polynomial equation entailment} problem decides whether $\forall
\vx. q(\vx) \implies q'(\vx)$ holds for all $\vx \in \bbfZ^{|\vx|}$.

Programs in single static assignment form can easily be transformed to
conjunctions of polynomial equations. We first show how to transform
statements into polynomial equations. An assignment statement is
translated to a polynomial equation with a single variable on the left
hand side. For a $\dslcode{Split}$ statement, it is transformed to an
equation with a linear expression of the assigned variables on the
left hand side (Algorithm~\ref{algorithm:polynomial-statements}). 
\begin{algorithm}
  \begin{algorithmic}[1]
    \Function{StatementToPolyEQ}{$s$}
    \Match{$s$}
      \Case{$v \leftarrow e$}
        \Return $v = e$
      \EndCase
      \Case{$\concat{v_h}{v_l} \leftarrow \dslcode{Split}(e, n)$}
        \Return $v_l + 2^n v_h = e$
      \EndCase
    \EndMatch
    \EndFunction
  \end{algorithmic}
  \caption{Polynomial Equation Transformation for Statements}
  \label{algorithm:polynomial-statements}
\end{algorithm}

A program is transformed to the conjunction of polynomial
equations corresponding to its statements
(Algorithm~\ref{algorithm:polynomial-programs}). Both algorithms are
specified straightforwardly in \coq.

\begin{algorithm}
  \begin{algorithmic}[1]
    \Function{ProgramToPolyEQ}{$p$}
    \Match{$p$}
      \Case{$\epsilon$} \Return $\top$ \EndCase
      \Case{$s; pp$}
        \Return \Call{StatementToPolyEQ}{$s$} $\wedge$
                \Call{ProgramToPolyEQ}{$pp$}
      \EndCase
    \EndMatch
    \EndFunction
  \end{algorithmic}
  \caption{Polynomial Equation Transformation for Programs}
  \label{algorithm:polynomial-programs}
\end{algorithm}

\begin{theorem}
  Let $p \in P$ be a well-formed program in static single assignment
  form. 
\todo{Explain polynomial equation is an abstraction. It admits all
  program behaviors and definitely more.}
\end{theorem}


Note that $\textsc{ProgramToPolyEQ}(p) \in Q$ for every $p \in P$.
Definition~\ref{definition:program-entailment} gives the modular
polynomial equation entailment problem corresponding to a program
specification.
\begin{definition}
  For any $q, q' \in Q$ and $p \in P$ in single static assignment
  form, define
  \begin{eqnarray*}
    \Pi(\hoaretriple{q}{p}{q'}) & \defn &
    \forall \vx \in \bbfZ^{|\vx|}. q(\vx) \wedge \varphi(\vx) \implies q'(\vx)
  \end{eqnarray*}
  where $\varphi(\vx) =
  \textsc{ProgramToPolyEQ}(p)$. 
  %$\Pi(\hoaretriple{q}{p}{q'})$ is called the modular polynomial
  %equation entailment problem for $\hoaretriple{q}{p}{q'}$. 
  \label{definition:program-entailment}
\end{definition}

Algorithm~\ref{algorithm:polynomial-programs} is only sound for
well-formed programs. 
Theorem~\ref{theorem:program-to-q-soundness} establishes the
soundnesss of our last transformation. Its proof is also certified in
\coq. 
\begin{theorem}
  \label{theorem:program-to-q-soundness}
  Let $q, q' \in Q$ be predicates, and $p \in P$ a program in single
  static assignment form and well-formed. 
  If $\Pi(\hoaretriple{q}{p}{q'})$ holds, then
  $\models$ $\hoaretriple{q}{p}{q'}$.
\end{theorem}



\subsubsection*{Summary of Translation}

Consider any predicates $q, q' \in Q$ and well-formed program $p \in
P$. Let $\theta_0 (v) = 0$ for every $v \in V$. By
Theorem~\ref{theorem:program-slicing}, \ref{theorem:ssa}, and
\ref{theorem:program-to-q-soundness}, we have the following deduction:
\begin{equation*}
  \begin{array}{cll}
    & \models \hoaretriple{q}{p}{q'}\\
    \Leftrightarrow
    & \models 
      \hoaretriple{q}{\textsc{SliceProgram}(\textsc{VarsInQ}(q'), p)}{q'}
    & \textmd{ (Theorem~\ref{theorem:program-slicing})}\\
    \Leftrightarrow
    & \models
      \hoaretriple{\textsc{SSAQ}(\theta_0, q)}
      {\hat{p}}
      {\textsc{SSAQ}(\hat{\theta}, q')}
    & \textmd{ (Theorem~\ref{theorem:ssa})}\\
    &
      \textmd{where } \langle \hat{\theta}, \hat{p} \rangle = 
      \textsc{SSAProgram} (\theta_0, \textsc{SliceProgram}
      (\textsc{VarsInQ}(q'), p))\\
    \Leftarrow
    & \Pi (\hoaretriple{\textsc{SSAQ}(\theta_0, q)}
      {\hat{p}}
      {\textsc{SSAQ}(\hat{\theta}, q')})
    & \textmd{ (Theorem~\ref{theorem:program-to-q-soundness})}
  \end{array}
\end{equation*}


\section{Solving Modular Polynomial Equation Entailment Problem}
\label{section:solving-algebraic-equations}

Consider the following system of $r + s$ (modulo) equations:
\begin{equation}
  \label{equation:modulo-polynomials}
  \begin{array}{rclcrcl}
    e_1 (\vec{x}) & = & e'_1 (\vec{x}) 
    & \hspace{.05\textwidth} &
    f_1 (\vec{x}) & \equiv & f'_1 (\vec{x}) \mod n_1\\
    & \vdots & & & & \vdots \\
    e_i (\vec{x}) & = & e'_i (\vec{x}) & &
    f_j (\vec{x}) & \equiv & f'_j (\vec{x}) \mod n_j\\
    & \vdots & & & & \vdots \\
    e_r (\vec{x}) & = & e'_r (\vec{x}) & &
    f_s (\vec{x}) & \equiv & f'_s (\vec{x}) \mod n_s
  \end{array}
\end{equation}
Our goal is to establish $g (\vec{x}) = g' (\vec{x})$ or $h (\vec{x})
\equiv h (\vec{x}) \mod m$ for every $\vec{x}$ satisfying the system
of equations~(\ref{equation:modulo-polynomials}). In order to prove it
in \coq, we first rewrite modulo equations $f_j (\vec{x}) \equiv f'_j
(\vec{x}) \mod n_j$ to existentially quantified equations $\exists d_j
\in \bbfZ, f_j (\vec{x}) = f'_j (\vec{x}) + n_j \times d_j$. In other
words, we want to show that for every $\vec{x} \in \bbfZ^{|\vec{x}|}$
satisfying 
\begin{equation}
  \label{equation:polynomials}
  \begin{array}{rclcrcl}
    e_1 (\vec{x}) & = & e'_1 (\vec{x}) 
    & \hspace{.05\textwidth} &
    f_1 (\vec{x}) & = & f'_1 (\vec{x}) + n_1 \times d_1 \\
    & \vdots & & & & \vdots \\
    e_i (\vec{x}) & = & e'_i (\vec{x}) & &
    f_j (\vec{x}) & = & f'_j (\vec{x}) + n_j \times d_j \\
    & \vdots & & & & \vdots \\
    e_r (\vec{x}) & = & e'_r (\vec{x}) & &
    f_s (\vec{x}) & = & f'_s (\vec{x}) + n_s \times d_s
  \end{array}
\end{equation}
with some $d_1, \ldots, d_s$, then $g (\vec{x}) = g' (\vec{x})$ or $h
(\vec{x}) \equiv h' (\vec{x}) \mod m$. There are two \coq tactics to
construct formal proofs. The tactic \dslcode{nsatz} is able to prove
equalities in integral domains. The tactic \dslcode{gbarith} is able
to prove modulo equalities in integral domains. We hence define a
\coq tactic which invokes \dslcode{nsatz} or \dslcode{gbarith} when
it tries to prove $g (\vec{x}) = g' (\vec{x})$ or $h (\vec{x}) \equiv
h' (\vec{x}) \mod m$ respectively.

\section{Evaluation}
\label{section:evaluation}



\subsection{Arithmetic Computation over $\bbfGF(2^{255}-19)$}
\label{subection:evaluation:multiplication}
 
The operator $\Gplus$ is defined by arithmetic computation over
$\bbfF$. In order to compute $p_0 \Gplus p_1$ for $p_0, p_1 \in G$,
arithmetic operations over $\bbfF$ need to be implemented. Recall that
an element in $\bbfF$ is represented by a 256-bit number. The na\"ive
implementation for arithmetic operations over $\bbfF$ would require
arithmetic computation over $\bbfZ$. Arithmetic computation for
255-bit integers however is not available in commodity computing
devices as of the year 2017. Long-integer arithmetic has to be carried
out by limbs of 32 or 64 bits depending on the underlying computer
architectures.

In practice, efficient long-integer arithmetic however is more
complicated. Consider subtracting a long integer from another. The
na\"ive implementation would simply subtract by limbs, propagate carry
flags, and add the prime number $\varrho$ if necessary. Carry flag
propagation however may induce waiting time on architectures with
multiple arithmetic logic units (ALUs) such as \todo{amd64? check!} 
More importantly, executime time of the na\"ive subtraction varies
when minuend is less than subtrahend. It allows timing attacks and is
insecure. The na\"ive implementation of 255-bit subtraction should
never be used in cryptographic primitives. 
Figure~\ref{figure:dsl:subtraction}
(Section~\ref{section:domain-specific-language}) in fact is an
efficient and secure implementation of 255-bit subtraction for the
AMD64 architecture. Notice that each limb can be computed in
parallel. It also has constant execution time. 

Long-integer multiplication is another interesting but much more
complicated operation. The na\"ive implementation for 255-bit
multiplication would compute a 510-bit result and then find the
corresponding 255-bit representation by a modulo operation. The
modulo operation however requires long-integer division. The na\"ive
implementation is very inefficient. An efficient implementation for
255-bit multiplication avoids division by performing modulo operations
aggressively. Observe that $2^{255} \Feq 19$ in $\bbfGF(\varrho)$. 
Whenever an intermediate result of the form $c \Ftimes 2^{255}$
is obtained, it is replaced by $c \Ftimes 19$. Recall that
elements in $\bbfF$ are represented by five limbs of 51-bit unsigned
integers. Consider multiplying two 255-bit values
\begin{equation*}
  \begin{array}{lcccccccccccl}
    x & = & \mathit{radix51} (x_4, x_3, x_2, x_1, x_0) & = &
            2^{51 \times 4} x_4 & + & 2^{51 \times 3} x_3 & + &
            2^{51 \times 2} x_2 & + & 2^{51 \times 1} x_1 & + &
            2^{51 \times 0} x_0 \textmd{ and}\\
    y & = & \mathit{radix51} (y_4, y_3, y_2, y_1, y_0) & = &
            2^{51 \times 4} y_4 & + & 2^{51 \times 3} y_3 & + &
            2^{51 \times 2} y_2 & + & 2^{51 \times 1} y_1 & + &
            2^{51 \times 0} y_0.
  \end{array}
\end{equation*}
The intermediate results $2^{255} x_4 y_1$, $2^{255} x_3 y_2$,
$2^{255} x_2 y_3$, and $2^{255} x_1 y_4$ can all be replaced by 
$19 x_4 y_1$, $19 x_3 y_2$, $19 x_2 y_3$, and $19 x_1 y_4$,
respectively. Division is not needed in such implementations.

We describe but a couple of optimizations in efficient implementations
of arithmetic operations over $\bbfF$. Real-world implementations
are necessarily optimized with various algebraic properties in
$\bbfF$. In our experiments, we took real-world efficient and secure
low-level implementations of arithmetic operations over
$\bbfGF(\varrho)$ in~\todo{reference?}, manually translated source
codes to our domain specific language, and specified their algebraic
properties, and performed certified verification with our new technique. 
Table~\ref{table:arithmetic-operations} summarizes the results.


\begin{table}[ht]
  \caption{Certified Verification of Arithmetic Operations over
    $\bbfGF(\varrho)$}
  \centering
  \begin{tabular}{|c|c|c|c|}
    \hline
             & number of lines & time (seconds) & remark\\
    \hline
    addition & 10 &                       & $a \Fplus b$ \\
    \hline
    subtraction & 15 &                    & $a \Fminus b$ \\
    \hline
    multiplication & 169 &                & $a \Ftimes b$\\
    \hline
    multiplication by 121666 & 31 &       & $121666 \Ftimes a$\\
    \hline
    square & 124 &                        & $a \Ftimes a$\\
    \hline
  \end{tabular}
  \label{table:arithmetic-operations}
\end{table}

\subsection{The Montgomery Ladderstep}
\label{subsection:evaluation:ladder-step}

\todo{Bo-Yin, please describe how the ladder step is used and why it is 
important for about .5 page. Maybe another .5 page for its mathematical 
properties? }

\begin{algorithm}[h]
\label{evaluation:ladder-step:montgomery}
\begin{algorithmic}[1]
\Function{ladderstep}{$x_1, x_2, z_2, x_3, z_3$}
\begin{multicols}{3}
\State $t_1 \leftarrow x_2 \Fplus z_2$
\State $t_2 \leftarrow x_2 \Fminus z_2$
\State $t_7 \leftarrow t_2 \Ftimes t_2$
\State $t_6 \leftarrow t_1 \Ftimes t_1$
\State $t_5 \leftarrow t_6 \Fminus t_7$
\State $t_3 \leftarrow x_3 \Fplus z_3$
\State $t_4 \leftarrow x_3 \Fminus z_3$\rule{0ex}{0ex}
\State $t_9 \leftarrow t_3 \Ftimes t_2$
\State $t_8 \leftarrow t_4 \Ftimes t_1$
\State $x_3 \leftarrow t_8 \Fplus t_9$
\State $z_3 \leftarrow t_8 \Fminus t_9$
\State $x_3 \leftarrow x_3 \Ftimes x_3$
\State $z_3 \leftarrow z_3 \Ftimes z_3$
\State $z_3 \leftarrow z_3 \Ftimes x_1$\rule{0ex}{0ex} 
\State $x_2 \leftarrow t_6 \Ftimes t_7$
\State $z_2 \leftarrow  121666 \Ftimes t_5$
\State $z_2 \leftarrow z_2 \Fplus t_7$
\State $z_2 \leftarrow z_2 \Ftimes t_5$
\State \Return $(x_2, z_2, x_3, z_3)$
\EndFunction
\end{multicols}
\end{algorithmic}
\caption{Montgomery Ladderstep}
\end{algorithm}

 
\todo{mention the coefficient $4$ in the ratio}

\section{Conclusion}
\label{section:conclusion}




\bibliographystyle{splncs-sorted}
\bibliography{refs}

\end{document}

%%% Local Variables: 
%%% mode: latex
%%% eval: (TeX-PDF-mode 1)
%%% TeX-master: "certified_vcg"
%%% End: 
