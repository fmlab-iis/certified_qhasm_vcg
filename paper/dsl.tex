
\begin{eqnarray*}
  N & ::= & \dslcode{1}\ |\ \dslcode{2}\ |\ \cdots\\
  C & ::= & \cdots \ |\ \dslcode{-2}\ |\ \dslcode{-1} \ |\ 0\ |\ 
            \dslcode{1}\ |\ \dslcode{2}\ |\ \cdots\\
  V & ::= & \dslcode{x} \ |\ \dslcode{y} \ |\ \dslcode{z} \ |\ \cdots\\
  E & ::= &  C \ |\ V \ |\  \dslcode{-}E \ |\ E \dslcode{+} E 
             \ |\ E \dslcode{-} E
             \ |\ E \times E \ |\ \dslcode{Pow} (E, N)
\end{eqnarray*}

\begin{eqnarray*}
  S & ::= & V \leftarrow E 
            \ |\  V, V \leftarrow \dslcode{Split} (E, N)\\
  P & ::= & \epsilon \ |\ S; P
\end{eqnarray*}

\begin{eqnarray*}
  Q & ::= & \top
     \ |\   E = E
     \ |\   E \equiv E \mod N
     \ |\   Q \wedge Q
\end{eqnarray*}

\begin{eqnarray*}
  H & ::= & \cond{Q} P \cond{Q}
\end{eqnarray*}


\begin{figure}
  \centering
  \[
  \begin{array}{lclcl}
    \begin{array}{rcl}
    \textmd{1:} && \dslcode{r}_0 \leftarrow \dslcode{x}_0; \\
    \textmd{2:} && \dslcode{r}_1 \leftarrow \dslcode{x}_1; \\
    \textmd{3:} && \dslcode{r}_2 \leftarrow \dslcode{x}_2; \\
    \textmd{4:} && \dslcode{r}_3 \leftarrow \dslcode{x}_3; \\
    \textmd{5:} && \dslcode{r}_5 \leftarrow \dslcode{x}_4; \\
    \end{array}
    &\hspace{.05\textwidth}&
    \begin{array}{rcl}
    \textmd{6:} && 
      \dslcode{r}_0 \leftarrow \dslcode{r}_0 + \dslcode{4503599627370458}; \\
    \textmd{7:} &&
      \dslcode{r}_1 \leftarrow \dslcode{r}_1 + \dslcode{4503599627370494}; \\
    \textmd{8:} &&
      \dslcode{r}_2 \leftarrow \dslcode{r}_2 + \dslcode{4503599627370494}; \\
    \textmd{9:} &&
      \dslcode{r}_3 \leftarrow \dslcode{r}_3 + \dslcode{4503599627370494}; \\
    \textmd{10:} && 
      \dslcode{r}_4 \leftarrow \dslcode{r}_4 + \dslcode{4503599627370494};\\
    \end{array}
    &\hspace{.05\textwidth}&
    \begin{array}{rcl}
    \textmd{11:} && \dslcode{r}_0 \leftarrow \dslcode{r}_0 - \dslcode{y}_0; \\
    \textmd{12:} && \dslcode{r}_1 \leftarrow \dslcode{r}_1 - \dslcode{y}_1; \\
    \textmd{13:} && \dslcode{r}_2 \leftarrow \dslcode{r}_2 - \dslcode{y}_2; \\
    \textmd{14:} && \dslcode{r}_3 \leftarrow \dslcode{r}_3 - \dslcode{y}_3; \\
    \textmd{15:} && \dslcode{r}_4 \leftarrow \dslcode{r}_4 - \dslcode{y}_4;
    \end{array}
  \end{array}
  \]
  \caption{Subtraction \dslcode{sub}$(\dslcode{x}_0, \dslcode{x}_1, 
    \dslcode{x}_2, \dslcode{x}_3, \dslcode{x}_4, \dslcode{y}_0,
    \dslcode{y}_1, \dslcode{y}_2, \dslcode{y}_3, \dslcode{y}_4)$}
  \label{figure:dsl:subtraction}
\end{figure}

In Figure~\ref{figure:dsl:subtraction}, a number in $\bbfGF(2^{255}-19)$ 
is represented by five 51-bit unsigned integers. The variables
$\dslcode{x}_0, \dslcode{x}_1, \dslcode{x}_2, \dslcode{x}_3,
\dslcode{x}_4$ for instance represent 
$\mathit{radix51}(\dslcode{x}_0, \dslcode{x}_1, \dslcode{x}_2,
\dslcode{x}_3, \dslcode{x}_4) \defn
\dslcode{x}_0 + 2^{51} \dslcode{x}_1 + 2^{102} \dslcode{x}_2 +
2^{153} \dslcode{x}_3 + 2^{204} \dslcode{x}_4$. The result of
subtraction is stored in the variables $\dslcode{r}_0, \dslcode{r}_1,
\dslcode{r}_2, \dslcode{r}_3, \dslcode{r}_4$. The specification of the
program is therefore
\[
\begin{array}{c}
\cond{\top}\\
\dslcode{sub}(\dslcode{x}_0, \dslcode{x}_1, \dslcode{x}_2,
  \dslcode{x}_3, \dslcode{x}_4, \dslcode{y}_0, \dslcode{y}_1,
  \dslcode{y}_2, \dslcode{y}_3, \dslcode{y}_4)\\
\cond{\mathit{radix}(\dslcode{x}_0, \dslcode{x}_1, \dslcode{x}_2,
\dslcode{x}_3, \dslcode{x}_4) -
\mathit{radix51}(\dslcode{y}_0, \dslcode{y}_1, \dslcode{y}_2,
\dslcode{y}_3, \dslcode{y}_4)
\equiv
\mathit{radix51}(\dslcode{r}_0, \dslcode{r}_1, \dslcode{r}_2,
\dslcode{r}_3, \dslcode{r}_4)
\mod 2^{255}-19
}.
\end{array}
\]