
X25519, the Curve25519 Elliptic Curve Diffie-Hellman (ECDH) key exchange
protocol, is a high-performance cryptosystem designed to use the
secure elliptic curve Curve25519.

Curve25519 is a curve offering 128 bits of security when used with
ECDH. Aside from allowing high-speed elliptic curve arithmetic, it is
also easier to implement properly, not covered by any known patents,
and less susceptible to implementation pitfalls such as weak
random-number generators. Its parameters were also selected using
easily described mathematical principles without resorting to any
random numbers or seeds, making it a good choice for those who
suspects that backdoors may have been inserted into other curves, such
as curves standardized by the U.S. National Institute of Standards and
Technology.

Because of its many superior qualities, currently Curve25519 is the de
facto alternative to the NIST P-256 curve and is used in a wide
variety of applications. For example, X25519 is the default key
exchange in the widely used OpenSSH software since 2014.

Most of the computation in X25519, in trade parlance, is in a
``variable base point multiplication'', and the centerpiece of most
such software is the ``Montgomery Ladderstep''. This is usually a
large constant-time assembly language program performing the
finite-field arithmetic that implements the mathematics on Curve25519.
Should the implementation of Montgomery ladderstep be incorrect, so
would that of X25519. Obviously for all its virtues, X25519 would be
pointless if the implementation is incorrect. This may be even more
relevant in cryptography than most of engineering, because crypto is
one of the few disciplines with the concept of an omnipresent
adversary, constantly looking for the smallest edge --- and hence
eager to trigger any normally unlikely event. Revising a cryptosystem
due to rare failures potentially leading to a cryptanalysis is not
unheard of \todo{Insert reference for the Proos paper on NTRU here}.
Thus, it is important for security that we can show the computations
comprising the Montgomery Ladderstep or (even better) the X25519
protocol to be correct. However, such verification is not easy due
the size both of the numbers in play (255 bits and more) and of the
program itself (10,000 machine instructions upwards).

Low-level implementations in assembly for arithmetic operations
are in fact available for the most efficient computation over large
Galois fields. A programming error in such implementations could
nullify cryptographic guarantees and compromise security. 

We have the following contributions:
\begin{itemize}
\item We propose a domain specific language for modeling low-level
  cryptographic programs over large finite fields.
\item We give a certified verification condition generator from
  programs written in our domain specific language to polynomial
  (modulo) equations over integral domains.
\item We verify arithmetic computation over a large finite field and a
  critical program (the Montgomery ladder step) automatically in a
  reasonable amount of time.  
\item To the best of our knowledge, our work is the first automatic
  and certified verification on real cryptographic programs with
  minimal human intervention.
\end{itemize}
