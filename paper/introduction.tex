
To utilize security offered by cryptography, cryptosystems must be
realized in computer programs. Such programs implement cryptosystems
through cryptographic primitives to compute mathematical functions. 
These cryptographic primitives are frequently invoked and manually
optimized for efficiency. Crucially, security guarantees of
cryptosystems depend very much on the correctness of these
cryptographic primitives. Programming errors in cryptographic
primitives can be exploited to compromise security of cryptosystems. 
It is of utmost importance to ensure the correctness of these
cryptographic primitives. In this paper, we develop a fully certified
technique to verify cryptographic primitives used in the security
protocol X25519 automatically.

X25519 is the Curve25519 Elliptic Curve Diffie-Hellman (ECDH) key
exchange protocol; it is a high-performance cryptosystem designed to
use the secure elliptic curve Curve25519. Curve25519 is an elliptic
curve offering 128 bits of security when used with ECDH. In addition
to allowing high-speed elliptic curve arithmetic, it is easier to
implement properly, not covered by any known patents, and moreover
less susceptible to implementation pitfalls such as weak 
random-number generators. Its parameters were also selected by
easily described mathematical principles without resorting to any
random numbers or seeds. These characteristics make Curve25519 a
preferred choice among curves suspicious of intentionally inserted
backdoors, such as curves standardized by the United States National
Institute of Standards and Technology (NIST). 
Indeed, Curve25519 is currently the
de facto alternative to the NIST P-256 curve. Consequently, X25519 has
a wide variety of applications including the default key exchange
protocol in \openssh since 2014.

Most of the computation in X25519, in trade parlance, is in a
``variable base point multiplication'', and the centerpiece of most
such software is the ``Montgomery Ladderstep''. This is usually a
large constant-time assembly language program performing the
finite-field arithmetic that implements the mathematics on Curve25519.
Should the implementation of Montgomery ladderstep be incorrect, so
would that of X25519. Obviously for all its virtues, X25519 would be
pointless if the implementation is incorrect. This may be even more
relevant in cryptography than most of engineering, because crypto is
one of the few disciplines with the concept of an omnipresent
adversary, constantly looking for the smallest edge --- and hence
eager to trigger any normally unlikely event. Revising a cryptosystem
due to rare failures potentially leading to a cryptanalysis is not
unheard of \todo{Insert reference for the Proos paper on NTRU here}.
Thus, it is important for security that we can show the computations
comprising the Montgomery Ladderstep or (even better) the X25519
protocol to be correct. However, such verification is not easy due
the size both of the numbers in play (255 bits and more) and of the
program itself (10,000 machine instructions upwards).

There are several obstacles to be overcome in the verification of
cryptographic primitives in Curve25519. The elliptic curve is defined
over the Galois field $\bbfGF(2^{255} - 19)$. Arithmetic computation
over the finite field needs to be correct. Particularly, non-linear
255-bit multiplication modulo $2^{255} - 19$ must be verified. Worse,
commodity computing devices do not support 255-bit arithmetic
computation directly. Arithmetic over the Galois field is implemented
by sequences of 32- or 64-bit arithmetic instructions of the underlying
architectures. One has to verify that a sequence of 32- or 64-bit
arithmetic instructions indeed performs, say, 255-bit multiplication
over the finite field. And this is but a single arithmetic computation
used in Curve25519. The Montgomery Ladderstep contains 18 arithmetic
operations over $\bbfGF(2^{255} - 19)$. It also needs to be verified
to benefit from security guarantees provided by Curve25519 and hence
the ECDH key exchange protocol X25519.

In this paper, we focus on algebraic properties of cryptographic
primitives. Cryptograhic primitives by their nature perform algebraic
computation. We aim to verify whether they perform intended algebraic
computation correctly. To this end, we propose the domain specific
language \mydsl for low-level implementations of cryptographic
primitives. Algebraic pre- and post-conditions of programs written in
\mydsl can subsequently be specified as Hoare triples. Such a
specification is converted to single static assignment form. It is
then translated into an algebraic problem (called the modular
polynomial equation entailment problem). We apply \coq tactics or the
computer algebra system \singular to solve the algebraic problem. To
reduce the complexity of the algebraic problem and hence improve the
efficiency of our technique, fragments irrelevant to algebraic
properties are removed by program slicing. The \coq proof assistant is
used to certify the correctness of all translations, as well as
solutions to algebraic problems derived from specifications.

We report case studies of verifying cryptographic primitives in the
X25519 ECDH key exchange protocol. For each arithmetic operations
(addition, multiplication, and square) over $\bbfGF(2^{255} - 19)$,
their low-level \qhasm programs are converted to our domain specific 
langauge \mydsl manually. Their specifications are then verified
automatically with our technique. The \qhasm implementation
of the Montgomery Ladderstep is formally verified similarly. 



We have the following contributions:
\begin{itemize}
\item We propose a domain specific language for modeling low-level
  cryptographic programs over large finite fields.
\item We give a certified verification condition generator from
  programs written in our domain specific language to polynomial
  (modulo) equations over integral domains.
\item We verify arithmetic computation over a large finite field and a
  critical program (the Montgomery ladder step) automatically in a
  reasonable amount of time.  
\item To the best of our knowledge, our work is the first automatic
  and certified verification on real cryptographic programs with
  minimal human intervention.
\end{itemize}
