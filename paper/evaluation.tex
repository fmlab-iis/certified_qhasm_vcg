
We evaluate our techniques in real-world low-level cryptographic
programs. In elliptic curve cryptography, arithmetic computation over
large finite fields is required for cryptographic primitives. For
instance, the elliptic curve $y^2 = x^3 + 486662 x^2 + x$ used in 
Curve25519 is over the Galois field $\bbfF = \bbfGF(\varrho)$ with
$\varrho = 2^{255} - 19$. The arithmetic operations (addition and
multiplication) in the curve are in fact over the finite field. To
make the finite field $\bbfF$ explicit, we may rewrite the elliptic
curve as the following equation: 
\begin{equation}
  \label{eq:curve25519}
  y \cdot_{\bbfF} y =_{\bbfF} x \cdot_{\bbfF} x \cdot_{\bbfF} x +_{\bbfF}
  486662 \cdot_{\bbfF} x \cdot_{\bbfF} x +_{\bbfF} x.
\end{equation}

Since arithmetic computation is over the Galois field and every
element of $\bbfGF(\varrho)$ can be represented by a 255-bit number,
any pair $(x, y)$ satisfying (\ref{eq:curve25519}) (called a
\emph{point} on the curve) can be represented by a pair of 255-bit
numbers. It can be shown that points on Curve25519 with the point at
infinity form a commutative group $\bbfG = (G, +_{\bbfG}, 0_{\bbfG})$
with $G = \{ (x, y) : x, y \textmd{ satisfying } (\ref{eq:curve25519})
\}$. Let $p_0 = (x_0, y_0), p_1 = (x_1, y_1) \in G$. We have $-p_0 =
(x_0, -y_0)$ and $p_0 +_{\bbfG} p_1 = (x, y)$ where
\begin{eqnarray*}
  m & = & (y_1 -_{\bbfF} y_0) /_{\bbfF} (x_1 -_{\bbfF} x_0)\\
  x & = & m \cdot_{\bbfF} m -_{\bbfF} x_0 -_{\bbfF} x_1\\
  y & = & m \cdot_{\bbfF} (x_0 -_{\bbfF} x) -_{\bbfF} y_0
     = m \cdot_{\bbfF} (x_1 -_{\bbfF} x) -_{\bbfF} y_1.
\end{eqnarray*}
The induced group $\bbfG$ plays an important role in elliptic
curve cryptography. It is essential to implement the commutative binary
operation $+_{\bbfG}$ very efficiently in practical cryptography. 