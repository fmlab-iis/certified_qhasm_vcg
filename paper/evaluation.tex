
We evaluate our techniques in real-world low-level cryptographic
programs. In elliptic curve cryptography, arithmetic computation over
large finite fields is required for cryptographic primitives. For
instance, the elliptic curve $y^2 = x^3 + 486662 x^2 + x$ used in 
Curve25519 is over the Galois field $\bbfF = \bbfGF(\varrho)$ with
$\varrho = 2^{255} - 19$. The arithmetic operations (addition and
multiplication) in the curve are in fact over the finite field. To
make the finite field $\bbfF$ explicit, we may rewrite the elliptic
curve in the following equation: 
\begin{equation}
  \label{eq:curve25519}
  y \cdot_{\bbfF} y =_{\bbfF} x \cdot_{\bbfF} x \cdot_{\bbfF} x +_{\bbfF}
  486662 \cdot_{\bbfF} x \cdot_{\bbfF} x +_{\bbfF} x.
\end{equation}

Since the curve is defined over $\bbfF$, arithmetic computation is
carried out over the Galois field. Note that every element of 
$\bbfGF(\varrho)$ can be represented by a 255-bit number. 255 bits
suffice to store results of any arithmetic computation. Particularly,
there is always a 255-bit number $c \in \bbfF$ such that $c =_{\bbfF} a
\cdot_{\bbfF} b$ for every $a, b \in \bbfF$. Points on
(\ref{eq:curve25519}) are thus represented by a pair of 255-bit numbers.