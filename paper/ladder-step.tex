
\hide{
\todo{Bo-Yin, please describe how the ladder step is used and why it is 
important for about .5 page. Maybe another .5 page for its mathematical 
properties? }
}

Recall that X25519 is based on the elliptic curve Curve25519. 
Cryptographic primitives of the key exchange protocol perform
sequences of algebraic operations on the Abelian group $\bbfG$ induced
by Curve25519, not on the finite field $\bbfGF(\varrho)$. As
aforementioned, the binary operation $\Gplus$ requires another
sequence of arithmetic computation over $\bbfGF(\varrho)$. Errors
could still be introduced or even implanted in implementations of
$\Gplus$. Correctness of arithmetic constructs over $\bbfGF(\varrho)$
does not necessarily imply the correctness of the mathematical
construct in X25519. The Montgomery Ladderstep is the mathematical
construct widely used to implement $\Gplus$ on Curve25519. We 
verify a low-level implementation of the mathematical construct in
this experiment. 


\begin{algorithm}[h]
\label{evaluation:ladder-step:montgomery}
\begin{algorithmic}[1]
\Function{ladderstep}{$x_1, z_2, x_2, z_2, x_3, z_3$}
\begin{multicols}{3}
\State $t_1 \leftarrow x_2 \Fplus z_2$
\State $t_2 \leftarrow x_2 \Fminus z_2$
\State $t_7 \leftarrow t_2 \Ftimes t_2$
\State $t_6 \leftarrow t_1 \Ftimes t_1$
\State $t_5 \leftarrow t_6 \Fminus t_7$
\State $t_3 \leftarrow x_3 \Fplus z_3$
\State $t_4 \leftarrow x_3 \Fminus z_3$\rule{0ex}{0ex}
\State $t_9 \leftarrow t_3 \Ftimes t_2$
\State $t_8 \leftarrow t_4 \Ftimes t_1$
\State $x_3 \leftarrow t_8 \Fplus t_9$
\State $z_3 \leftarrow t_8 \Fminus t_9$
\State $x_3 \leftarrow x_3 \Ftimes x_3$
\State $z_3 \leftarrow z_3 \Ftimes z_3$
\State $z_3 \leftarrow z_3 \Ftimes x_1$\rule{0ex}{0ex} 
\State $x_2 \leftarrow t_6 \Ftimes t_7$
\State $z_2 \leftarrow  121666 \Ftimes t_5$
\State $z_2 \leftarrow z_2 \Fplus t_7$
\State $z_2 \leftarrow z_2 \Ftimes t_5$
\State \Return $(x_2, z_2, x_3, z_3)$
\EndFunction
\end{multicols}
\end{algorithmic}
\caption{Montgomery Ladderstep}
\end{algorithm}

 
\todo{mention the coefficient $4$ in the ratio}